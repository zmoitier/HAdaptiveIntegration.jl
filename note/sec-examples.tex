% !TeX root = main.tex
% LTeX: language=en-US

%***===========================================================
\section{Example of embedded quadrature rule}

%***-----------------------------------------------------------
\subsection{Reference \texorpdfstring{\(n\)}{n}-simplex}

\begin{lemma}
    The moments on the reference \(n\)-simplex \( \Simplex_\Rf^n \) are
    \[
        \int_{\Simplex_\Rf^n} \vec{x}^{\vec{\alpha}} \di{\vec{x}} = \frac{\vec{\alpha}!}{\plr*{\abs{\vec{\alpha}}_1 + n}!},
    \]
    for \( \vec{\alpha} \) multi-index.
\end{lemma}
\begin{proof}
    \todo{Find a reference.}
\end{proof}

One point cubature of order 1:
\[
    \vec{x}_* = \frac{1}{n+1} \vec{1}
    \quad \text{and} \quad
    w_* = \frac{1}{n!}
\]

\( x_*, x_0, x_1, \ldots, x_n \) where
\begin{align*}
    x_0 & = \plr*{0, \ldots, 0}
    \\
    x_1 & = \plr*{1, 0, \ldots, 0}
    \\
        & \vdots
    \\
    x_n & = \plr*{0, \ldots, 0, 1}
\end{align*}

\begin{align*}
    w_* + (n+1) w_1               & = \frac{1}{n!}
    \\
    \frac{w_*}{n+1} + w_1         & = \frac{1}{(n+1)!}
    \\
    \frac{w_*}{\plr{n+1}^2} + w_1 & = \frac{2}{(n+2)!}
    \\
    \frac{w_*}{\plr{n+1}^2}       & = \frac{1}{(n+2)!}
\end{align*}

\begin{align*}
    w_* & = \frac{\plr{n+1}^2}{\plr{n+2}!}
    \\
    w_1 & = \frac{1}{\plr{n+2}!}
\end{align*}

%***-----------------------------------------------------------
\subsection{Reference \texorpdfstring{\(n\)}{n}-orthotope}

\begin{lemma}
    Moments:
    \[
        \int_{\Orthotope_\Rf^n} \vec{x}^{\vec{\alpha}} \di{\vec{x}} = \prod_{i=1}^n \frac{1}{\alpha_i + 1}
    \]
\end{lemma}

\[
    \vec{x}_0 = \frac{1}{2} \vec{1}
\]

One point cubature of order 1:
\[
    \vec{x}_* = \frac{1}{2} \vec{1}
    \quad \text{and} \quad
    w_* = 1
\]

\( x_*, x_0, x_1, \ldots, x_{2^n-1} \) where
\begin{align*}
    x_i = \plr{\text{binary representation}}
\end{align*}

\begin{align*}
    w_* + 2^n w_1               & = 1
    \\
    \frac{w_*}{2} + 2^{n-1} w_1 & = \frac{1}{2}
    \\
    \frac{w_*}{4} + 2^{n-1} w_1 & = \frac{1}{3}
    \\
    \frac{w_*}{4} + 2^{n-2} w_1 & = \frac{1}{4}
\end{align*}

\begin{align*}
    w_* + 2^n w_1     & = 1
    \\
    w_* + 2^{n+1} w_1 & = \frac{4}{3}
\end{align*}

\begin{align*}
    w_* & = \frac{2}{3}
    \\
    w_1 & = \frac{1}{2^n 3}
\end{align*}
