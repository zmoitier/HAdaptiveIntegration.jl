% !TeX root = main.tex
% LTeX: language=en-US

\documentclass[english,12pt]{article}


%*** \Usepackages =============================================
\usepackage[utf8]{inputenc}
\usepackage[T1]{fontenc}
\usepackage[a4paper,margin=2.5cm]{geometry}
\usepackage{amssymb,amsthm,mathtools}
\usepackage{graphicx}
\usepackage{babel}
\usepackage[colorlinks]{hyperref}
\usepackage{microtype}
% \usepackage{caption}
% \usepackage{subcaption}
% \usepackage{enumitem}
% \usepackage{tabularx}

%*** Links and drawing ----------------------------------------
% \usepackage{xcolor}
\usepackage{tikz}

%*** Math -----------------------------------------------------
\usepackage{siunitx}
\usepackage[capitalize,nameinlink]{cleveref}
\numberwithin{equation}{section}
\raggedbottom{}
\allowdisplaybreaks{}

%*** Bibliography ---------------------------------------------
\usepackage[backend=biber,style=alphabetic,sorting=nyt]{biblatex}
\addbibresource{references.bib}
\usepackage{csquotes}


%*** Colors ===================================================
%*** \definecolor ---------------------------------------------
\definecolor{Blue}{HTML}{1F77B4}
\definecolor{Orange}{HTML}{FF7F0E}
\definecolor{Green}{HTML}{2CA02C}
\definecolor{Red}{HTML}{D62728}
\definecolor{Grey}{HTML}{7F7F7F}
% \definecolor{Purple}{HTML}{9467BD}
% \definecolor{Brown}{HTML}{8C564B}
% \definecolor{Pink}{HTML}{E377C2}

\NewDocumentCommand{\Black}{m}{\textcolor{black}{#1}}
\NewDocumentCommand{\Blue}{m}{\textcolor{Blue}{#1}}
\NewDocumentCommand{\Orange}{m}{\textcolor{Orange}{#1}}
\NewDocumentCommand{\Green}{m}{\textcolor{Green}{#1}}
\NewDocumentCommand{\Red}{m}{\textcolor{Red}{#1}}
\NewDocumentCommand{\Grey}{m}{\textcolor{Grey}{#1}}
% \NewDocumentCommand{\Purple}{m}{\textcolor{Purple}{#1}}
% \NewDocumentCommand{\Brown}{m}{\textcolor{Brown}{#1}}
% \NewDocumentCommand{\Pink}{m}{\textcolor{Pink}{#1}}

\NewDocumentCommand{\todo}{m}{\noindent{\color{Red}\texttt{:\!:todo:\!:} \bfseries #1}}
\NewDocumentCommand{\zmcom}{m}{\noindent{\color{Green}\texttt{:\!:Z:\!:} #1}}

%*** hypersetup ----------------------------------------------
\hypersetup{%
    colorlinks = True,
    linkcolor  = Red,
    citecolor  = Green,
    urlcolor   = Blue,
}


%*** Math =====================================================
%*** mathbb ---------------------------------------------------
\NewDocumentCommand{\bbC}{}{\mathbb{C}}
\NewDocumentCommand{\bbN}{}{\mathbb{N}}
\NewDocumentCommand{\bbR}{}{\mathbb{R}}
\NewDocumentCommand{\bbZ}{}{\mathbb{Z}}

%*** mathcal --------------------------------------------------
\NewDocumentCommand{\calT}{}{\mathcal{T}}

%*** Constants ------------------------------------------------
\NewDocumentCommand{\ex}{}{\mathsf{e}}
\NewDocumentCommand{\im}{}{\mathsf{i}\mkern1mu}

%*** Spaces ---------------------------------------------------
\NewDocumentCommand{\spC}{}{\mathscr{C}}
\NewDocumentCommand{\spL}{}{\mathrm{L}}
\NewDocumentCommand{\spH}{}{\mathrm{H}}

%*** Keywords -------------------------------------------------
\NewDocumentCommand{\loc}{}{\mathrm{loc}}
\NewDocumentCommand{\comp}{}{\mathrm{comp}}

%*** Operators ------------------------------------------------
\NewDocumentCommand{\di}{m}{\mathop{}\!\mathrm{d}#1}
\DeclareMathOperator{\OO}{\mathcal{O}}
\DeclareMathOperator{\oo}{\mathcal{\scriptstyle{}O}}

%*** Delimiter ------------------------------------------------
\DeclarePairedDelimiter{\plr}\lparen\rparen%
\DeclarePairedDelimiter{\clr}\lbrack\rbrack%
\DeclarePairedDelimiter{\blr}\lbrace\rbrace%
\DeclarePairedDelimiter{\alr}\langle\rangle%
\DeclarePairedDelimiter{\abs}\lvert\rvert%
\DeclarePairedDelimiter{\norm}\lVert\rVert%

\DeclarePairedDelimiterXPP\normF[1]{}\lVert\rVert{_\mathrm{F}}{#1}

\DeclarePairedDelimiterX\setst[2]\lbrace\rbrace{#1\:\delimsize\vert\:\mathopen{}#2} % chktex 21
\DeclarePairedDelimiterX\setwt[2]\lbrace\rbrace{#1\:{:}\:\mathopen{}#2}

\DeclarePairedDelimiterX\ioo[2]\lparen\rparen{#1,\:#2}
\DeclarePairedDelimiterX\ioc[2]\lparen\rbrack{#1,\:#2}
\DeclarePairedDelimiterX\ico[2]\lbrack\rparen{#1,\:#2}
\DeclarePairedDelimiterX\icc[2]\lbrack\rbrack{#1,\:#2}

\NewDocumentCommand\restr{s m m}{%
    \IfBooleanTF{#1}{%
        \left. #2 \right\vert_{#3}%
    }{%
        #2 \arrowvert_{#3}%
    }%
}

%*** Specials -------------------------------------------------
\NewDocumentCommand{\eps}{}{\varepsilon}

\NewDocumentCommand{\Simplex}{}{\Delta}
\NewDocumentCommand{\St}{}{\mathsf{st}}
\NewDocumentCommand{\Rf}{}{\mathsf{ref}}


%*** Environment ==============================================
\theoremstyle{definition}
\newtheorem{definition}{Definition}[section]
\newtheorem{notation}[definition]{Notation}
\newtheorem{assumption}[definition]{Assumption}

\crefname{assumption}{Assumption}{Assumptions}
\Crefname{assumption}{Assumption}{Assumptions}

\theoremstyle{plain}
\newtheorem{lemma}[definition]{Lemma}
\newtheorem{theorem}[definition]{Theorem}
\newtheorem{corollary}[definition]{Corollary}

\theoremstyle{remark}
\newtheorem{remark}[definition]{Remark}
\newtheorem{example}[definition]{Example}


%*** Title ====================================================
\title{Note on adaptive integration on simplices}

\author{
    Luiz \textsc{Faria}\({}^1\)
    \and
    Zo{\"\i}s \textsc{Moitier}\({}^2\)
}

\date{\raggedright\footnotesize%
    \({^1}\)POEMS, CNRS, Inria, ENSTA Paris, Institut Polytechnique de Paris, 91120 Palaiseau, France.\\
    \({}^2\)IDEFIX, Inria, ENSTA Paris, Institut Polytechnique de Paris, 91120 Palaiseau, France.\\[1em]
    \large\Red{\textbf{\today}}
}

%*** Document =================================================
\begin{document}

\maketitle

% \begin{abstract}
%     \ldots
% \end{abstract}

% \setcounter{tocdepth}{2}
% \tableofcontents


%***===========================================================
\section{Mathematical settings}

For \( n  \in \bbN \), an \( n \)-simplex is an \( n \)-dimensional polytope which is the convex hull of its \( n+1 \) vertices \( v_0, \ldots, v_n \in \bbR^n \).
We assume that the \( n+1 \) vertices are independent, which means that the \( n \) vectors \( \blr{v_1 - v_0, \ldots, v_n - v_0} \) are linearly independent.

\begin{definition}
    We define two special simplices.
    \begin{itemize}
        \item The \emph{standard \( n \)-simplex} (which is also a regular polytope):
              \[
                  \Simplex_\St^n \coloneqq \setst*{\lambda \in \bbR_+^{n+1}}{\lambda_0 + \cdots + \lambda_n = 1} \subset \bbR^{n+1},
              \]
              which is also defined by being the convex-hull of the \( n+1 \) canonical basis vectors (\( \plr{1,0,\ldots, 0} \), \ldots) in \( \bbR^{n+1} \).

        \item The \emph{reference \( n \)-simplex}:
              \[
                  \Simplex_\Rf^n \coloneqq \setst*{u \in \bbR_+^n}{u_1 + \cdots + u_n \leq 1} \subset \bbR^n,
              \]
              which is also defined by being the convex-hull of the vector \( \plr{0, \ldots, 0} \) and of the \( n \) canonical basis vectors (\( \plr{1,0,\ldots, 0} \), \ldots) in \( \bbR^n \).
    \end{itemize}
    We drop the superscript \( n \) when it is clear from the context.
\end{definition}

Let \( \Simplex \) be an \( n \)-simplex defined by the points \( v_0, \ldots, v_n \) and \( \Phi_\Simplex \colon \Simplex_\Rf \to \Simplex \) the affine map from the reference simplex to the simplex \( \Simplex \) defined by \( \Phi_\Simplex \colon u \mapsto A_\Simplex u + v_0 \) where
\[
    A_\Simplex = \begin{pmatrix}
        v_1 - v_0 & \cdots & v_n - v_0
    \end{pmatrix}.
\]
\begin{align*}
    \int_\Simplex f(x) \di{x}
     & = \int_{\Simplex_\Rf} f\plr*{\Phi_\Simplex(u)} \ \abs*{\det A_\Simplex} \di{u},
    \\
     & = \abs*{\det A_\Simplex} \int_{\Simplex_\Rf} f\plr*{\Phi_\Simplex(u)} \ \abs*{\det A_\Simplex} \di{u},
    \\
     & \approx \abs*{\det A_\Simplex} \sum_{i=1}^N w_i f\plr*{\Phi_\Simplex(u_i)},
\end{align*}


%***===========================================================
% \section{Acknowledgement}


%***===========================================================
% \section{Funding}


%*** Bibliography =============================================
% \printbibliography%[heading=bibintoc,title={References}]


%*** Appendix =================================================
% \appendix

%***===========================================================
% \section{Miscellaneous}

\end{document}
