% JuliaCon proceedings template
\documentclass{juliacon}
\setcounter{page}{1}

\newcommand{\bx}{\boldsymbol{x}}

\begin{document}

% **************GENERATED FILE, DO NOT EDIT**************

\title{\texttt{HAdaptiveIntegration.jl}: Adaptive numerical integration over simplices and orthotopes}

\author[1]{Luiz Faria}
\author[2]{Zoïs Moitier}
\affil[1]{POEMS, CNRS, Inria, ENSTA, Institut Polytechnique de Paris, 91120 Palaiseau, France}
\affil[2]{Inria, Unité de Mathématiques Appliquées, ENSTA, Institut Polytechnique de Paris, 91120 Palaiseau, France}

\keywords{Julia, Scientific computing, Numerical integration}

\hypersetup{
pdftitle = {\texttt{HAdaptiveIntegration.jl}: Adaptive numerical integration over simplices and orthotopes},
pdfsubject = {JuliaCon 2025 Proceedings},
pdfauthor = {Luiz Faria, Zoïs Moitier},
pdfkeywords = {Julia, Scientific computing, Numerical integration},
}



\maketitle

\begin{abstract}

	This is a guide for authors who are preparing papers for JuliaCon using the \LaTeX{} document
	preparation system and the \verb|juliacon| class file.

\end{abstract}

\section{Introduction}

\texttt{HAdaptiveIntegration.jl} is a Julia package designed for numerical integration over
multidimensional domains. It approximates integrals of the form
%
\[
	I = \int_{\Omega} f(\bx) \, \mathrm{d}\bx
\]
%
where
\begin{itemize}
	\item $f : \mathbb{R}^d \to \mathbb{F}$ is any Julia function mapping $d$-dimensional
	      vectors to elements of type $\mathbb{F}$ supporting multiplication by a real scalar,
	      addition and a norm (i.e. a normed real vector space).
	\item $\Omega \subset \mathbb{R}^d$ is the integration domain (simplices and
	      orthotopes)
\end{itemize}

The package employs an adaptive approach, dynamically refining the integration domain as
needed. It uses embedded cubature rules to provide error estimates, aiming to achieve high
accuracy while minimizing function evaluations.

% Features include:

% - Adaptive integration over simplices and orthotope of **any dimension**,
% - Utilization of **efficient tabulated cubatures** for low-dimensional simplices and
% orthotopes,
% - Support for custom embedded cubature rules,
% - Arbitrary precision arithmetic.

\section{Usage}
\label{sec:basic-usage}


\textbf{Create a domain:}

\textbf{Integrate:}

\subsection{Creating a domain}
\label{subsec:label}

This is a test

\section{Mathematical background}
\label{sec:math-background}

\subsection{Embedded cubatures}

- Two quadratures

- A posteriori error estimate

\subsection{Adaptive algorithm}
- Subdivision of domains

- Selection of domains to refine

- Stopping criteria

\section{Complexity estimates}

\subsection{Uniform regularity}

\subsection{Isotropic (nearly-)singularities}

Things like


\section{Benchmarks}

\section{Extended precision}
\label{sec:extended-precision}





\input{bib.tex}

\end{document}

% Inspired by the International Journal of Computer Applications template

%%% Local Variables:
%%% mode: LaTeX
%%% TeX-master: t
%%% End:
